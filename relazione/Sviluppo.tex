% -------------------------------- SVILUPPO -------------------------------------

\chapter{Sviluppo}
\section{Testing automatizzato}

Il testing automatizzato è un requisito di qualunque progetto software che si rispetti, e consente di verificare che non vi siano regressioni nelle funzionalità a fronte di aggiornamenti.
%
Per quanto riguarda questo progetto è considerato sufficiente un test minimale, a patto che sia completamente automatico.
%
Test che richiedono l'intervento da parte dell'utente sono considerati \textit{negativamente} nel computo del punteggio finale.

\subsection*{Elementi positivi}

\begin{itemize}
	\item Si descrivono molto brevemente i componenti che si è deciso di sottoporre a test automatizzato.
	\item Si utilizzano suite specifiche (e.g. JUnit) per il testing automatico.
	\item Se sono stati eseguiti test manuali di rilievo, si elencano descrivendo brevemente la ragione per cui non sono stati automatizzati. Ad esempio, se tutto il team sviluppa e testa su uno stesso sistema operativo e si sono svolti test manuali per verificare, ad esempio, il corretto funzionamento dell'interfaccia grafica o di librerie native su altri sistemi operativi, può avere senso menzionare la cosa.
\end{itemize}

\subsection*{Elementi negativi}
\begin{itemize}
	\item Non si realizza alcun test automatico.
	\item La non presenza di testing viene aggravata dall'adduzione di motivazioni non valide. Ad esempio, si scrive che l'interfaccia grafica non è testata automaticamente perché è \emph{impossibile} farlo\footnote{Testare in modo automatico le interfacce grafiche è possibile (si veda, come esempio, \url{https://github.com/TestFX/TestFX}), semplicemente nel corso non c'è modo e tempo di introdurvi questo livello di complessità. Il fatto che non vi sia stato insegnato come farlo non implica che sia impossibile!}.
	\item Si descrive un testing di tipo manuale in maniera prolissa.
	\item Si descrivono test effettuati manualmente che sarebbero potuti essere automatizzati, ad esempio scrivendo che si è usata l'applicazione manualmente.
	\item Si descrivono test non presenti nei sorgenti del progetto.
	\item I test, quando eseguiti, falliscono.
\end{itemize}

\section{Metodologia di lavoro}

Ci aspettiamo, leggendo questa sezione, di trovare conferma alla divisione operata nella sezione del design di dettaglio, e di capire come è stato svolto il lavoro di integrazione.
%
\textbf{Andrà realizzata una sotto-sezione separata per ciascuno studente} che identifichi le porzioni di progetto sviluppate, separando quelle svolte in autonomia da quelle sviluppate in collaborazione.
%
Diversamente dalla sezione di design, in questa è consentito elencare package/classi, se lo studente ritiene sia il modo più efficace di convogliare l'informazione.
%
Si ricorda che l'impegno deve giustificare circa 40-50 ore di sviluppo (è normale e fisiologico che approssimativamente la metà del tempo sia impiegata in analisi e progettazione).

\subsection*{Elementi positivi}

\begin{itemize}
	\item Si identifica con precisione il ruolo di ciascuno all'interno del gruppo, ossia su quale parte del progetto ciascuno dei componenti si è concentrato maggiormente.
	\item La divisione dei compiti è equa, ossia non vi sono membri del gruppo che hanno svolto molto più lavoro di altri.
	\item La divisione dei compiti è coerente con quanto descritto nelle parti precedenti della relazione.
	\item La divisione dei compiti è realistica, ossia le dipendenze fra le parti sviluppate sono minime.
	\item Si identifica quale parte del software è stato sviluppato da tutti i componenti insieme.
	\item Si spiega in che modo si sono integrate le parti di codice sviluppate separatamente, evidenziando eventuali problemi. Ad esempio, una strategia è convenire sulle interfacce da usare (ossia, occuparsi insieme di stabilire l'architettura) e quindi procedere indipendentemente allo sviluppo di parti differenti. Una possibile problematica potrebbe essere una dimenticanza in fase di design architetturale che ha costretto ad un cambio e a modifiche in fase di integrazione. Una situazione simile è la norma nell'ingegneria di un sistema software non banale, ed il processo di progettazione top-down con raffinamento successivo è il così detto processo ``a spirale''.
	\item Si descrive in che modo è stato impiegato il DVCS.
\end{itemize}

\subsection*{Elementi negativi}
\begin{itemize}
	\item Non si chiarisce chi ha fatto cosa.
	\item C'è discrepanza fra questa sezione e le sezioni che descrivono il design dettagliato.
	\item Tutto il progetto è stato svolto lavorando insieme invece che assegnando una parte a ciascuno.
	\item Non viene descritta la metodologia di integrazione delle parti sviluppate indipendentemente.
	\item Uso superficiale del DVCS.
\end{itemize}

\section{Note di sviluppo}

Questa sezione, come quella riguardante il design dettagliato va svolta \textbf{singolarmente da ogni membro del gruppo}.

Ciascuno dovrà mettere in evidenza eventuali particolarità del suo metodo di sviluppo, ed in particolare:
\begin{itemize}
	\item \textbf{Elencare} (fare un semplice elenco per punti, non un testo!) le feature \textit{avanzate} del linguaggio e dell'ecosistema Java che sono state
	utilizzate. Le feature di interesse sono:
	\begin{itemize}
		\item Progettazione con generici, ad esempio costruzione di nuovi tipi generici, e uso di generici bounded. Uso di classi generiche di libreria non è considerato avanzato.
		\item Uso di lambda expressions
		\item Uso di \texttt{Stream}, di \texttt{Optional} o di altri costrutti funzionali
		\item Uso della reflection
		\item Definizione ed uso di nuove annotazioni
		\item Uso del Java Platform Module System
		\item Uso di parti di libreria non spiegate a lezione (networking, compressione, parsing XML, eccetera...)
		\item Uso di librerie di terze parti (incluso JavaFX): Google Guava, Apache Commons...
		\item Uso di build systems
	\end{itemize}
	Si faccia molta attenzione a non scrivere banalità, elencando qui features di tipo ``core'', come le eccezioni, le enumerazioni, o le inner class: nessuna di queste è considerata avanzata.
	\item Descrivere \textit{molto brevemente} le librerie utilizzate nella propria parte di progetto, se non trattate a lezione (ossia, se librerie di terze parti e/o se componenti del JDK non visti, come le socket). Si ricorda che l'utilizzo di librerie è valutato \emph{positivamente}.
	\item Sviluppo di algoritmi particolarmente interessanti \emph{non forniti da alcuna libreria} (spesso può convenirvi chiedere sul forum se ci sia una libreria per fare una certa cosa, prima di gettarvi a capofitto per scriverla voi stessi).
\end{itemize}
%
In questa sezione, \textit{dopo l'elenco}, è anche bene evidenziare eventuali pezzi di codice ``riadattati'' (o scopiazzati...) da Internet o da altri progetti, pratica che tolleriamo ma che non raccomandiamo.
%
I pattern di design, invece \textbf{non} vanno messi qui.
%
L'uso di pattern di design (come suggerisce il nome) è un aspetto avanzato di design, non di implementazione,
e non va in questa sezione.

\subsection*{Elementi positivi}

\begin{itemize}
	\item Si elencano gli aspetti avanzati di linguaggio che sono stati impiegati
	\item Si elencano le librerie che sono state utilizzate
	\item Si descrivono aspetti particolarmente complicati o rilevanti relativi all'implementazione,
	ad esempio, in un'applicazione performance critical, un uso particolarmente avanzato di meccanismi
	di caching, oppure l'implementazione di uno specifico algoritmo.
	\item Se si è utilizzato un particolare algoritmo, se ne cita la fonte originale. Ad esempio, se
	si è usato Mersenne Twister per la generazione dei numeri pseudo-random, si cita \cite{mersenne}.
	\item Si identificano parti di codice prese da altri progetti, dal web, o comunque scritte in forma originale da altre persone. In tal senso, si ricorda che agli ingegneri non è richiesto di re-inventare la ruota continuamente: se si cita debitamente la sorgente è tollerato fare uso di di snippet di codice per risolvere velocemente problemi non banali. Nel caso in cui si usino snippet di codice di qualità discutibile, oltre a menzionarne l'autore originale si invitano gli studenti ad adeguare tali parti di codice agli standard e allo stile del progetto. Contestualmente, si fa presente che è largamente meglio fare uso di una libreria che copiarsi pezzi di codice: qualora vi sia scelta (e tipicamente c'è), si preferisca la prima via.
\end{itemize}

\subsection*{Elementi negativi}
\begin{itemize}
	\item Si elencano feature core del linguaggio invece di quelle segnalate. Esempi di feature
	core da non menzionare sono:
	\begin{itemize}
		\item eccezioni;
		\item classi innestate;
		\item enumerazioni;
		\item interfacce.
	\end{itemize}
	\item Si elencano applicazioni di terze parti (peggio se per usarle occorre licenza, e lo
	studente ne è sprovvisto) che non c'entrano nulla con lo sviluppo, ad esempio:
	\begin{itemize}
		\item Editor di grafica vettoriale come Inkscape o Adobe Illustrator;
		\item Editor di grafica scalare come GIMP o Adobe Photoshop;
		\item Editor di audio come Audacity;
		\item Strumenti di design dell'interfaccia grafica come SceneBuilder: il codice è in ogni caso inteso come sviluppato da voi.
	\end{itemize}
	\item Si descrivono aspetti di scarsa rilevanza, o si scende in dettagli inutili.
	\item Sono presenti parti di codice sviluppate originalmente da altri che non vengono
	debitamente segnalate. In tal senso, si ricorda agli studenti che i docenti hanno accesso a tutti i
	progetti degli anni passati, a Stack Overflow, ai principali blog di sviluppatori ed esperti Java (o sedicenti tali), ai blog dedicati allo sviluppo di soluzioni e applicazioni (inclusi blog dedicati ad Android e allo sviluppo di videogame), nonché ai social network. Conseguentemente, è \emph{molto} conveniente \emph{citare} una fonte ed usarla invece di tentare di spacciare per proprio il lavoro di altri.
	\item Si elencano design pattern
\end{itemize}