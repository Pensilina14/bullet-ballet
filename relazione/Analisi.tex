% ------------------------------------ ANALISI ----------------------------------

\chapter{Analisi}

In questo capitolo andrà fatta l'analisi dei requisiti e quella del problema, ossia verranno elencate le cose che l'applicazione dovrà fare (requisiti) e verrà descritto il dominio applicativo (analisi del problema).
%
In fase di analisi, è molto importante tenere a mente che non vi deve essere alcun riferimento al design né tantomeno alle tecnologie implementative, ovvero, non si deve indicare come il software sarà internamente realizzato.
%
La fase di analisi, infatti, \textit{precede} qualunque azione di design o di implementazione.

\section{Requisiti}

Nell'analisi dei \emph{requisiti} dell'applicazione si dovrà spiegare cosa l'applicazione dovrà fare.
%
Non ci si deve concentrare sui particolari problemi, ma esclusivamente su cosa si desidera che l'applicazione faccia.
%
È consigliato descrivere separatamente i requisiti funzionali (quelli che descrivono l'effettivo
comportamento dell'applicazione) da quelli non funzionali (requisiti che non riguardano direttamente
aspetti comportamentali, come sicurezza, performance, eccetera).

\subsection*{Elementi positivi}
\begin{itemize}
	\item Si fornisce una descrizione in linguaggio naturale di ciò che il software dovrà fare.
	\item Gli obiettivi sono spiegati con chiarezza, per punti.
	\item Se il software è stato commissionato o è destinato ad un utente o compagnia specifici, il committente viene nominato.
	\item Se vi sono termini il cui significato non è immediatamente intuibile, essi vengono spiegati.
	\item Vengono descritti separatamente requisiti funzionali e non funzionali.
	\item Considerato a un paio di pagine un limite ragionevole alla lunghezza della parte sui requisiti, in quello spazio si deve cercare di chiarire \textit{tutti} gli aspetti dell'applicazione, non lasciando decisioni che impattano la parte ``esterna'' alla discussione del design (che dovrebbe solo occuparsi della parte ``interna'').
\end{itemize}

\subsection*{Elementi negativi}
\begin{itemize}
	\item Si forniscono indicazioni circa le soluzioni che si vogliono adottare
	\item Si forniscono dettagli di tipo tecnico o implementativo (parlando di classi, linguaggi di programmazione, librerie, eccetera)
\end{itemize}

\subsection*{Esempio}
Il software, commissionato dal gestore del centro di ricerca ``Aperture Laboratories Inc.''\footnote{\url{http://aperturescience.com/}}, mira alla costruzione di una intelligenza artificiale di nome GLaDOS (Genetic Lifeform and Disk Operating System).
%
Per intelligenza artificiale si intende un software in grado di assumere decisioni complesse in maniera semi autonoma sugli argomenti di sua competenza, a partire dai vincoli e dagli obiettivi datigli dall'utente.

\subsubsection{Requisiti funzionali}
\begin{itemize}
	\item La suddetta intelligenza artificiale dovrà occuparsi di coordinare le attività all'interno
	delle camere di test di Aperture, guidando l'utente attraverso un certo numero di sfide di
	difficoltà crescente. Una camera di test è un ambiente realizzato da Aperture Laboratories Inc. al
	fine di mettere alla prova le proprie tecnologie di manipolazione dell'ambiente. All'interno della
	camera di test, un soggetto qualificato è incaricato di sfruttare gli strumenti messi a
	disposizione da Aperture per risolvere alcuni rompicapi. I rompicapi sono di tipo fisico (ad
	esempio, manipolazione di oggetti, pressione di pulsanti, azionamento di leve), e si ritengono
	conclusi una volta che il soggetto riesce a trovare l'uscita dalla camera di test.
	\item Il piano preciso ed il numero delle sfide sarà variabile, e GLaDOS dovrà essere in grado di adattarsi dinamicamente e di fornire indicazioni di guida.
	\item La personalità di GLaDOS dovrà essere modificabile.
	\item GLaDOS dovrà essere in grado di comunicare col reparto cucina di Aperture, per ordinare torte da donare agli utenti che completassero l'ultima camera di test con successo.
\end{itemize}

\subsubsection{Requisiti non funzionali}
\begin{itemize}
	\item GLaDOS dovrà essere estremamente efficiente nell'uso delle risorse. Le specifiche tecniche parlano della possibilità di funzionare su dispositivi alimentati da una batteria a patata.
\end{itemize}

\section{Analisi e modello del dominio}

In questa sezione si descrive il modello del \textit{dominio
	applicativo}, descrivendo le \textit{entità} in gioco ed i rapporti fra loro.
%
Si possono sollevare eventuali aspetti particolarmente impegnativi, descrivendo perché lo sono, senza inserire idee circa possibili soluzioni, ovvero sull'organizzazione interna del software.
%
Infatti, la fase di analisi va effettuata \textbf{prima} del progetto: né il progetto né il software esistono nel momento in cui si effettua l'analisi.
%
La discussione di aspetti propri del software (ossia, della \textit{soluzione} al problema e non del problema stesso) appartengono alla sfera della progettazione, e vanno discussi successivamente.

È obbligatorio fornire uno schema UML del dominio, che diventerà anche lo scheletro della
parte ``entity'' del modello dell'applicazione, ovvero degli elementi costitutivi del modello (in ottica MVC - Model View Controller): se l'analisi è ben fatta, dovreste ottenere una gerarchia di concetti che rappresentano le entità che compongono il problema da risolvere.
%
Un'analisi ben svolta \textbf{prima} di cimentarsi con lo sviluppo rappresenta un notevole aiuto per
le fasi successive: è sufficiente descrivere a parole il dominio, quindi estrarre i sostantivi
utilizzati, capire il loro ruolo all'interno del problema, le relazioni che intercorrono fra loro, e
reificarli in interfacce.

\subsection*{Elementi positivi}
\begin{itemize}
	\item Viene descritto accuratamente il modello del dominio.
	\item Alcuni problemi, se non risolubili in assoluto o nel monte ore, vengono dichiarati come problemi che non saranno risolti o sarano risolti in futuro.
	\item Si modella il dominio in forma di UML, descrivendolo appropriatamente.
\end{itemize}

\subsection*{Elementi negativi}
\begin{itemize}
	\item Manca una descrizione a parole del modello del dominio.
	\item Manca una descrizione UML delle entità del dominio e delle relazioni che intercorrono fra loro.
	\item Vengono elencate soluzioni ai problemi, invece della descrizione degli stessi.
	\item Vengono presentati elementi di design, o peggio aspetti implementativi.
	\item Viene mostrato uno schema UML che include elementi implementativi o non utili alla descrizione del dominio, ma volti alla soluzione (non devono vedersi, ad esempio, campi o metodi privati, o cose che non siano equivalenti ad interfacce).
\end{itemize}

\subsection*{Esempio}
GLaDOS dovrà essere in grado di accedere ad un'insieme di camere di test.
%
Tale insieme di camere prende il nome di percorso.
%
Ciascuna camera è composta di challenge successivi.
%
GLaDOS è responsabile di associare a ciascun challenge un insieme di consigli (suggestions) destinati all'utente (subject), dipendenti da possibili eventi.
%
GLaDOS dovrà poter comunicare coi locali cucina per approntare le torte.
%
Le torte potranno essere dolci, oppure semplici promesse di dolci che verranno disattese.

Gli elementi costitutivi il problema sono sintetizzati in \Cref{img:analysis}.

La difficoltà primaria sarà quella di riuscire a correlare lo stato corrente dell'utente e gli eventi in modo tale da generare i corretti suggerimenti.
%
Questo richiederà di mettere in campo appropriate strategie di intelligenza artificiale.

Data la complessità di elaborare consigli via AI senza intervento umano, la prima versione del software fornita prevederà una serie di consigli forniti dall'utente.

Il requisito non funzionale riguardante il consumo energetico richiederà studi specifici sulle performance di GLaDOS che non potranno essere effettuati all'interno del monte ore previsto: tale feature sarà oggetto di futuri lavori.

\begin{figure}[h]
	\centering{}
	\includegraphics{img/analysis.pdf}
	\caption{Schema UML dell'analisi del problema, con rappresentate le entità principali ed i rapporti fra loro}
	\label{img:analysis}
\end{figure}